\documentclass[]{article}
\usepackage{lmodern}
\usepackage{amssymb,amsmath}
\usepackage{ifxetex,ifluatex}
\usepackage{fixltx2e} % provides \textsubscript
\ifnum 0\ifxetex 1\fi\ifluatex 1\fi=0 % if pdftex
  \usepackage[T1]{fontenc}
  \usepackage[utf8]{inputenc}
\else % if luatex or xelatex
  \ifxetex
    \usepackage{mathspec}
  \else
    \usepackage{fontspec}
  \fi
  \defaultfontfeatures{Ligatures=TeX,Scale=MatchLowercase}
\fi
% use upquote if available, for straight quotes in verbatim environments
\IfFileExists{upquote.sty}{\usepackage{upquote}}{}
% use microtype if available
\IfFileExists{microtype.sty}{%
\usepackage{microtype}
\UseMicrotypeSet[protrusion]{basicmath} % disable protrusion for tt fonts
}{}
\usepackage[margin=1in]{geometry}
\usepackage{hyperref}
\hypersetup{unicode=true,
            pdftitle={assignment4},
            pdfauthor={Keiryn Hart, 300428418},
            pdfborder={0 0 0},
            breaklinks=true}
\urlstyle{same}  % don't use monospace font for urls
\usepackage{color}
\usepackage{fancyvrb}
\newcommand{\VerbBar}{|}
\newcommand{\VERB}{\Verb[commandchars=\\\{\}]}
\DefineVerbatimEnvironment{Highlighting}{Verbatim}{commandchars=\\\{\}}
% Add ',fontsize=\small' for more characters per line
\usepackage{framed}
\definecolor{shadecolor}{RGB}{248,248,248}
\newenvironment{Shaded}{\begin{snugshade}}{\end{snugshade}}
\newcommand{\AlertTok}[1]{\textcolor[rgb]{0.94,0.16,0.16}{#1}}
\newcommand{\AnnotationTok}[1]{\textcolor[rgb]{0.56,0.35,0.01}{\textbf{\textit{#1}}}}
\newcommand{\AttributeTok}[1]{\textcolor[rgb]{0.77,0.63,0.00}{#1}}
\newcommand{\BaseNTok}[1]{\textcolor[rgb]{0.00,0.00,0.81}{#1}}
\newcommand{\BuiltInTok}[1]{#1}
\newcommand{\CharTok}[1]{\textcolor[rgb]{0.31,0.60,0.02}{#1}}
\newcommand{\CommentTok}[1]{\textcolor[rgb]{0.56,0.35,0.01}{\textit{#1}}}
\newcommand{\CommentVarTok}[1]{\textcolor[rgb]{0.56,0.35,0.01}{\textbf{\textit{#1}}}}
\newcommand{\ConstantTok}[1]{\textcolor[rgb]{0.00,0.00,0.00}{#1}}
\newcommand{\ControlFlowTok}[1]{\textcolor[rgb]{0.13,0.29,0.53}{\textbf{#1}}}
\newcommand{\DataTypeTok}[1]{\textcolor[rgb]{0.13,0.29,0.53}{#1}}
\newcommand{\DecValTok}[1]{\textcolor[rgb]{0.00,0.00,0.81}{#1}}
\newcommand{\DocumentationTok}[1]{\textcolor[rgb]{0.56,0.35,0.01}{\textbf{\textit{#1}}}}
\newcommand{\ErrorTok}[1]{\textcolor[rgb]{0.64,0.00,0.00}{\textbf{#1}}}
\newcommand{\ExtensionTok}[1]{#1}
\newcommand{\FloatTok}[1]{\textcolor[rgb]{0.00,0.00,0.81}{#1}}
\newcommand{\FunctionTok}[1]{\textcolor[rgb]{0.00,0.00,0.00}{#1}}
\newcommand{\ImportTok}[1]{#1}
\newcommand{\InformationTok}[1]{\textcolor[rgb]{0.56,0.35,0.01}{\textbf{\textit{#1}}}}
\newcommand{\KeywordTok}[1]{\textcolor[rgb]{0.13,0.29,0.53}{\textbf{#1}}}
\newcommand{\NormalTok}[1]{#1}
\newcommand{\OperatorTok}[1]{\textcolor[rgb]{0.81,0.36,0.00}{\textbf{#1}}}
\newcommand{\OtherTok}[1]{\textcolor[rgb]{0.56,0.35,0.01}{#1}}
\newcommand{\PreprocessorTok}[1]{\textcolor[rgb]{0.56,0.35,0.01}{\textit{#1}}}
\newcommand{\RegionMarkerTok}[1]{#1}
\newcommand{\SpecialCharTok}[1]{\textcolor[rgb]{0.00,0.00,0.00}{#1}}
\newcommand{\SpecialStringTok}[1]{\textcolor[rgb]{0.31,0.60,0.02}{#1}}
\newcommand{\StringTok}[1]{\textcolor[rgb]{0.31,0.60,0.02}{#1}}
\newcommand{\VariableTok}[1]{\textcolor[rgb]{0.00,0.00,0.00}{#1}}
\newcommand{\VerbatimStringTok}[1]{\textcolor[rgb]{0.31,0.60,0.02}{#1}}
\newcommand{\WarningTok}[1]{\textcolor[rgb]{0.56,0.35,0.01}{\textbf{\textit{#1}}}}
\usepackage{longtable,booktabs}
\usepackage{graphicx,grffile}
\makeatletter
\def\maxwidth{\ifdim\Gin@nat@width>\linewidth\linewidth\else\Gin@nat@width\fi}
\def\maxheight{\ifdim\Gin@nat@height>\textheight\textheight\else\Gin@nat@height\fi}
\makeatother
% Scale images if necessary, so that they will not overflow the page
% margins by default, and it is still possible to overwrite the defaults
% using explicit options in \includegraphics[width, height, ...]{}
\setkeys{Gin}{width=\maxwidth,height=\maxheight,keepaspectratio}
\IfFileExists{parskip.sty}{%
\usepackage{parskip}
}{% else
\setlength{\parindent}{0pt}
\setlength{\parskip}{6pt plus 2pt minus 1pt}
}
\setlength{\emergencystretch}{3em}  % prevent overfull lines
\providecommand{\tightlist}{%
  \setlength{\itemsep}{0pt}\setlength{\parskip}{0pt}}
\setcounter{secnumdepth}{0}
% Redefines (sub)paragraphs to behave more like sections
\ifx\paragraph\undefined\else
\let\oldparagraph\paragraph
\renewcommand{\paragraph}[1]{\oldparagraph{#1}\mbox{}}
\fi
\ifx\subparagraph\undefined\else
\let\oldsubparagraph\subparagraph
\renewcommand{\subparagraph}[1]{\oldsubparagraph{#1}\mbox{}}
\fi

%%% Use protect on footnotes to avoid problems with footnotes in titles
\let\rmarkdownfootnote\footnote%
\def\footnote{\protect\rmarkdownfootnote}

%%% Change title format to be more compact
\usepackage{titling}

% Create subtitle command for use in maketitle
\providecommand{\subtitle}[1]{
  \posttitle{
    \begin{center}\large#1\end{center}
    }
}

\setlength{\droptitle}{-2em}

  \title{assignment4}
    \pretitle{\vspace{\droptitle}\centering\huge}
  \posttitle{\par}
    \author{Keiryn Hart, 300428418}
    \preauthor{\centering\large\emph}
  \postauthor{\par}
      \predate{\centering\large\emph}
  \postdate{\par}
    \date{29/06/2020}

\usepackage{booktabs}
\usepackage{longtable}
\usepackage{array}
\usepackage{multirow}
\usepackage{wrapfig}
\usepackage{float}
\usepackage{colortbl}
\usepackage{pdflscape}
\usepackage{tabu}
\usepackage{threeparttable}
\usepackage{threeparttablex}
\usepackage[normalem]{ulem}
\usepackage{makecell}
\usepackage{xcolor}

\begin{document}
\maketitle

\begin{Shaded}
\begin{Highlighting}[]
\CommentTok{# Load the "readxl" package to read in data from an Excel file.}
\KeywordTok{library}\NormalTok{(readxl)}

\CommentTok{# Read in the heart disease dataset.}
\NormalTok{hd <-}\StringTok{ }\KeywordTok{read_xlsx}\NormalTok{(}\StringTok{"C:/Users/Keiryn Hart/Documents/Uni/Data 303/Heart Disease.xlsx"}\NormalTok{, }\DataTypeTok{sheet =} \StringTok{"Data"}\NormalTok{, }\DataTypeTok{na =} \StringTok{"NA"}\NormalTok{)}
\end{Highlighting}
\end{Shaded}

Question 1: a)

\begin{Shaded}
\begin{Highlighting}[]
\KeywordTok{ncol}\NormalTok{(hd)}
\end{Highlighting}
\end{Shaded}

\begin{verbatim}
## [1] 16
\end{verbatim}

\begin{Shaded}
\begin{Highlighting}[]
\NormalTok{k <-}\StringTok{ }\DecValTok{16}
\NormalTok{i <-}\StringTok{ }\DecValTok{1}
\NormalTok{fr.prop <-}\StringTok{ }\KeywordTok{data.frame}\NormalTok{(}\KeywordTok{c}\NormalTok{(}\StringTok{"Frequency (n)"}\NormalTok{, }\StringTok{"Proportion (p)"}\NormalTok{))}
\KeywordTok{colnames}\NormalTok{(fr.prop) <-}\StringTok{ "Variable"}
\ControlFlowTok{for}\NormalTok{(i }\ControlFlowTok{in} \DecValTok{1}\OperatorTok{:}\NormalTok{k)\{}
\NormalTok{  count <-}\StringTok{ }\KeywordTok{sum}\NormalTok{(}\KeywordTok{is.na}\NormalTok{(hd[,i]))}
\NormalTok{  prop <-}\StringTok{ }\NormalTok{count}\OperatorTok{/}\KeywordTok{nrow}\NormalTok{(hd)}
\NormalTok{  results <-}\StringTok{ }\KeywordTok{data.frame}\NormalTok{(}\KeywordTok{c}\NormalTok{(count, prop))}
\NormalTok{  fr.prop <-}\StringTok{ }\KeywordTok{cbind}\NormalTok{(fr.prop, results)}
  \KeywordTok{colnames}\NormalTok{(fr.prop)[i}\OperatorTok{+}\DecValTok{1}\NormalTok{] <-}\StringTok{ }\KeywordTok{paste}\NormalTok{(}\StringTok{"VARIABLE"}\NormalTok{, i, }\DataTypeTok{sep=} \StringTok{"_"}\NormalTok{)}
\NormalTok{\}}

\NormalTok{fr.prop}
\end{Highlighting}
\end{Shaded}

\begin{verbatim}
##         Variable VARIABLE_1 VARIABLE_2   VARIABLE_3 VARIABLE_4
## 1  Frequency (n)          0          0 105.00000000          0
## 2 Proportion (p)          0          0   0.02476415          0
##     VARIABLE_5 VARIABLE_6 VARIABLE_7 VARIABLE_8 VARIABLE_9 VARIABLE_10
## 1 29.000000000    53.0000          0          0          0 50.00000000
## 2  0.006839623     0.0125          0          0          0  0.01179245
##   VARIABLE_11 VARIABLE_12  VARIABLE_13  VARIABLE_14  VARIABLE_15
## 1           0           0 19.000000000 1.0000000000 388.00000000
## 2           0           0  0.004481132 0.0002358491   0.09150943
##   VARIABLE_16
## 1           0
## 2           0
\end{verbatim}

\begin{enumerate}
\def\labelenumi{\alph{enumi})}
\setcounter{enumi}{1}
\item
\end{enumerate}

\begin{Shaded}
\begin{Highlighting}[]
\NormalTok{hd.complete <-}\StringTok{ }\KeywordTok{na.omit}\NormalTok{(hd)}
\NormalTok{missing <-}\StringTok{ }\KeywordTok{nrow}\NormalTok{(hd) }\OperatorTok{-}\StringTok{ }\KeywordTok{nrow}\NormalTok{(hd.complete)}
\NormalTok{missing}\OperatorTok{/}\KeywordTok{nrow}\NormalTok{(hd)}
\end{Highlighting}
\end{Shaded}

\begin{verbatim}
## [1] 0.1372642
\end{verbatim}

\begin{enumerate}
\def\labelenumi{\alph{enumi})}
\setcounter{enumi}{2}
\item
\end{enumerate}

\begin{Shaded}
\begin{Highlighting}[]
\NormalTok{hd.complete[, }\StringTok{"SBP_CAT"}\NormalTok{] <-}\StringTok{ }\OtherTok{NA}
\NormalTok{a <-}\StringTok{ }\KeywordTok{nrow}\NormalTok{(hd.complete)}
\ControlFlowTok{for}\NormalTok{(i }\ControlFlowTok{in} \DecValTok{1}\OperatorTok{:}\NormalTok{a)\{}
  \ControlFlowTok{if}\NormalTok{(hd.complete}\OperatorTok{$}\NormalTok{SBP[i] }\OperatorTok{<}\StringTok{ }\DecValTok{120}\NormalTok{)\{}
\NormalTok{    hd.complete}\OperatorTok{$}\NormalTok{SBP_CAT[i] <-}\StringTok{ "normal"}
\NormalTok{  \}}
  \ControlFlowTok{else} \ControlFlowTok{if}\NormalTok{(hd.complete}\OperatorTok{$}\NormalTok{SBP[i] }\OperatorTok{>}\StringTok{ }\DecValTok{119} \OperatorTok{&}\StringTok{ }\NormalTok{hd.complete}\OperatorTok{$}\NormalTok{SBP[i] }\OperatorTok{<}\StringTok{ }\DecValTok{130}\NormalTok{)\{}
\NormalTok{    hd.complete}\OperatorTok{$}\NormalTok{SBP_CAT[i] <-}\StringTok{ "elevated"}
\NormalTok{  \}}
  \ControlFlowTok{else} \ControlFlowTok{if}\NormalTok{(hd.complete}\OperatorTok{$}\NormalTok{SBP[i] }\OperatorTok{>}\StringTok{ }\DecValTok{129} \OperatorTok{&}\StringTok{ }\NormalTok{hd.complete}\OperatorTok{$}\NormalTok{SBP[i] }\OperatorTok{<}\StringTok{ }\DecValTok{140}\NormalTok{)\{}
\NormalTok{    hd.complete}\OperatorTok{$}\NormalTok{SBP_CAT[i] <-}\StringTok{ "High Stage 1"}
\NormalTok{  \}}
  \ControlFlowTok{else} \ControlFlowTok{if}\NormalTok{(hd.complete}\OperatorTok{$}\NormalTok{SBP[i] }\OperatorTok{>}\StringTok{ }\DecValTok{139} \OperatorTok{&}\StringTok{ }\NormalTok{hd.complete}\OperatorTok{$}\NormalTok{SBP[i] }\OperatorTok{<}\StringTok{ }\DecValTok{180}\NormalTok{)\{}
\NormalTok{    hd.complete}\OperatorTok{$}\NormalTok{SBP_CAT[i] <-}\StringTok{ "High Stage 2"}
\NormalTok{  \}}
  \ControlFlowTok{else}\NormalTok{\{}
\NormalTok{    hd.complete}\OperatorTok{$}\NormalTok{SBP_CAT[i] <-}\StringTok{ "hypertensive crisis"}
\NormalTok{  \}}
\NormalTok{\}}

\NormalTok{hd.complete}
\end{Highlighting}
\end{Shaded}

\begin{verbatim}
## # A tibble: 3,658 x 17
##      SEX   AGE  EDUC SMOKER   CIG BP_MED STROKE HYPER  DIAB  CHOL   SBP
##    <dbl> <dbl> <dbl>  <dbl> <dbl>  <dbl>  <dbl> <dbl> <dbl> <dbl> <dbl>
##  1     1    39     4      0     0      0      0     0     0   195  106 
##  2     0    46     2      0     0      0      0     0     0   250  121 
##  3     1    48     1      1    20      0      0     0     0   245  128.
##  4     0    61     3      1    30      0      0     1     0   225  150 
##  5     0    46     3      1    23      0      0     0     0   285  130 
##  6     0    43     2      0     0      0      0     1     0   228  180 
##  7     0    63     1      0     0      0      0     0     0   205  138 
##  8     0    45     2      1    20      0      0     0     0   313  100 
##  9     1    52     1      0     0      0      0     1     0   260  142.
## 10     1    43     1      1    30      0      0     1     0   225  162 
## # ... with 3,648 more rows, and 6 more variables: DBP <dbl>, BMI <dbl>,
## #   HR <dbl>, GLUC <dbl>, HD_RISK <dbl>, SBP_CAT <chr>
\end{verbatim}

\begin{enumerate}
\def\labelenumi{\alph{enumi})}
\setcounter{enumi}{3}
\tightlist
\item
  because it can provide meaninful qualitative differences where the cut
  off points for the different levels of the predictor can reflect
  better levels for the model predictions.
\end{enumerate}

Question 2:

\begin{enumerate}
\def\labelenumi{\alph{enumi})}
\item
\end{enumerate}

\begin{Shaded}
\begin{Highlighting}[]
\NormalTok{logistic <-}\StringTok{  }\KeywordTok{glm}\NormalTok{(HD_RISK }\OperatorTok{~}\StringTok{ }\NormalTok{SBP }\OperatorTok{+}\StringTok{ }\NormalTok{DBP }\OperatorTok{+}\StringTok{ }\KeywordTok{factor}\NormalTok{(SEX) }\OperatorTok{+}\StringTok{ }\NormalTok{AGE }\OperatorTok{+}\StringTok{ }\KeywordTok{factor}\NormalTok{(EDUC) }\OperatorTok{+}\StringTok{ }\NormalTok{CIG }\OperatorTok{+}\StringTok{ }\NormalTok{CHOL }\OperatorTok{+}\StringTok{ }\NormalTok{BMI }\OperatorTok{+}\StringTok{ }\NormalTok{GLUC, }\DataTypeTok{family =}\NormalTok{ binomial, }\DataTypeTok{data =}\NormalTok{ hd.complete)}
\KeywordTok{summary}\NormalTok{(logistic)}
\end{Highlighting}
\end{Shaded}

\begin{verbatim}
## 
## Call:
## glm(formula = HD_RISK ~ SBP + DBP + factor(SEX) + AGE + factor(EDUC) + 
##     CIG + CHOL + BMI + GLUC, family = binomial, data = hd.complete)
## 
## Deviance Residuals: 
##     Min       1Q   Median       3Q      Max  
## -1.9444  -0.5969  -0.4262  -0.2843   2.9063  
## 
## Coefficients:
##                Estimate Std. Error z value Pr(>|z|)    
## (Intercept)   -8.962559   0.579182 -15.475  < 2e-16 ***
## SBP            0.018249   0.003484   5.238 1.63e-07 ***
## DBP           -0.002798   0.006385  -0.438   0.6612    
## factor(SEX)1   0.544414   0.108953   4.997 5.83e-07 ***
## AGE            0.063433   0.006695   9.474  < 2e-16 ***
## factor(EDUC)2 -0.188485   0.123171  -1.530   0.1259    
## factor(EDUC)3 -0.196924   0.149848  -1.314   0.1888    
## factor(EDUC)4 -0.052129   0.164125  -0.318   0.7508    
## CIG            0.019463   0.004191   4.644 3.42e-06 ***
## CHOL           0.002371   0.001127   2.105   0.0353 *  
## BMI            0.006714   0.012639   0.531   0.5952    
## GLUC           0.007186   0.001674   4.293 1.77e-05 ***
## ---
## Signif. codes:  0 '***' 0.001 '**' 0.01 '*' 0.05 '.' 0.1 ' ' 1
## 
## (Dispersion parameter for binomial family taken to be 1)
## 
##     Null deviance: 3121.2  on 3657  degrees of freedom
## Residual deviance: 2758.8  on 3646  degrees of freedom
## AIC: 2782.8
## 
## Number of Fisher Scoring iterations: 5
\end{verbatim}

\begin{Shaded}
\begin{Highlighting}[]
\KeywordTok{library}\NormalTok{(car)}
\end{Highlighting}
\end{Shaded}

\begin{verbatim}
## Warning: package 'car' was built under R version 3.6.3
\end{verbatim}

\begin{verbatim}
## Loading required package: carData
\end{verbatim}

\begin{Shaded}
\begin{Highlighting}[]
\KeywordTok{library}\NormalTok{(kableExtra)}
\end{Highlighting}
\end{Shaded}

\begin{verbatim}
## Warning: package 'kableExtra' was built under R version 3.6.3
\end{verbatim}

\begin{Shaded}
\begin{Highlighting}[]
\KeywordTok{kable}\NormalTok{(}\KeywordTok{vif}\NormalTok{(logistic), }\DataTypeTok{digits =} \DecValTok{2}\NormalTok{, }\DataTypeTok{caption =} \StringTok{"VIF Values"}\NormalTok{)}\OperatorTok
\StringTok{  }\KeywordTok{kable_styling}\NormalTok{()}
\end{Highlighting}
\end{Shaded}

\begin{table}[t]

\caption{\label{tab:unnamed-chunk-7}VIF Values}
\centering
\begin{tabular}{l|r|r|r}
\hline
  & GVIF & Df & GVIF\textasciicircum{}(1/(2*Df))\\
\hline
SBP & 3.02 & 1 & 1.74\\
\hline
DBP & 2.78 & 1 & 1.67\\
\hline
factor(SEX) & 1.24 & 1 & 1.11\\
\hline
AGE & 1.29 & 1 & 1.13\\
\hline
factor(EDUC) & 1.10 & 3 & 1.02\\
\hline
CIG & 1.24 & 1 & 1.11\\
\hline
CHOL & 1.06 & 1 & 1.03\\
\hline
BMI & 1.18 & 1 & 1.09\\
\hline
GLUC & 1.02 & 1 & 1.01\\
\hline
\end{tabular}
\end{table}

none of the vif values exceed 1.74 and all are clearly far below 10
which indicates that there is little evidence relationg to
mulitcoliniarity of the predictors and therefore there is no need to
remove any of the predictors.

\begin{enumerate}
\def\labelenumi{\alph{enumi})}
\setcounter{enumi}{1}
\item
\end{enumerate}

regression equation:

\#log(p\textsuperscript{/1-p}) = HD\_RISK + 0.0183\emph{SBP -
0.00280}DBP + 0.544\emph{factor(SEX)1 + 0.0634}AGE -
0.189\emph{factor(SEX)2 - 0.197}factor(SEX)3 - 0.0521\emph{factor(SEX)4
+ 0.195}CIG + 0.00237\emph{CHOL + 0.00671}BMI + 0.00719*GLUC

\begin{Shaded}
\begin{Highlighting}[]
\KeywordTok{library}\NormalTok{(pander)}
\end{Highlighting}
\end{Shaded}

\begin{verbatim}
## Warning: package 'pander' was built under R version 3.6.3
\end{verbatim}

\begin{Shaded}
\begin{Highlighting}[]
\KeywordTok{pander}\NormalTok{(}\KeywordTok{summary}\NormalTok{(logistic))}
\end{Highlighting}
\end{Shaded}

\begin{longtable}[]{@{}ccccc@{}}
\toprule
\begin{minipage}[b]{0.24\columnwidth}\centering
~\strut
\end{minipage} & \begin{minipage}[b]{0.14\columnwidth}\centering
Estimate\strut
\end{minipage} & \begin{minipage}[b]{0.16\columnwidth}\centering
Std. Error\strut
\end{minipage} & \begin{minipage}[b]{0.12\columnwidth}\centering
z value\strut
\end{minipage} & \begin{minipage}[b]{0.14\columnwidth}\centering
Pr(\textgreater{}\textbar{}z\textbar{})\strut
\end{minipage}\tabularnewline
\midrule
\endhead
\begin{minipage}[t]{0.24\columnwidth}\centering
\textbf{(Intercept)}\strut
\end{minipage} & \begin{minipage}[t]{0.14\columnwidth}\centering
-8.963\strut
\end{minipage} & \begin{minipage}[t]{0.16\columnwidth}\centering
0.5792\strut
\end{minipage} & \begin{minipage}[t]{0.12\columnwidth}\centering
-15.47\strut
\end{minipage} & \begin{minipage}[t]{0.14\columnwidth}\centering
5.157e-54\strut
\end{minipage}\tabularnewline
\begin{minipage}[t]{0.24\columnwidth}\centering
\textbf{SBP}\strut
\end{minipage} & \begin{minipage}[t]{0.14\columnwidth}\centering
0.01825\strut
\end{minipage} & \begin{minipage}[t]{0.16\columnwidth}\centering
0.003484\strut
\end{minipage} & \begin{minipage}[t]{0.12\columnwidth}\centering
5.238\strut
\end{minipage} & \begin{minipage}[t]{0.14\columnwidth}\centering
1.626e-07\strut
\end{minipage}\tabularnewline
\begin{minipage}[t]{0.24\columnwidth}\centering
\textbf{DBP}\strut
\end{minipage} & \begin{minipage}[t]{0.14\columnwidth}\centering
-0.002798\strut
\end{minipage} & \begin{minipage}[t]{0.16\columnwidth}\centering
0.006385\strut
\end{minipage} & \begin{minipage}[t]{0.12\columnwidth}\centering
-0.4382\strut
\end{minipage} & \begin{minipage}[t]{0.14\columnwidth}\centering
0.6612\strut
\end{minipage}\tabularnewline
\begin{minipage}[t]{0.24\columnwidth}\centering
\textbf{factor(SEX)1}\strut
\end{minipage} & \begin{minipage}[t]{0.14\columnwidth}\centering
0.5444\strut
\end{minipage} & \begin{minipage}[t]{0.16\columnwidth}\centering
0.109\strut
\end{minipage} & \begin{minipage}[t]{0.12\columnwidth}\centering
4.997\strut
\end{minipage} & \begin{minipage}[t]{0.14\columnwidth}\centering
5.83e-07\strut
\end{minipage}\tabularnewline
\begin{minipage}[t]{0.24\columnwidth}\centering
\textbf{AGE}\strut
\end{minipage} & \begin{minipage}[t]{0.14\columnwidth}\centering
0.06343\strut
\end{minipage} & \begin{minipage}[t]{0.16\columnwidth}\centering
0.006695\strut
\end{minipage} & \begin{minipage}[t]{0.12\columnwidth}\centering
9.474\strut
\end{minipage} & \begin{minipage}[t]{0.14\columnwidth}\centering
2.685e-21\strut
\end{minipage}\tabularnewline
\begin{minipage}[t]{0.24\columnwidth}\centering
\textbf{factor(EDUC)2}\strut
\end{minipage} & \begin{minipage}[t]{0.14\columnwidth}\centering
-0.1885\strut
\end{minipage} & \begin{minipage}[t]{0.16\columnwidth}\centering
0.1232\strut
\end{minipage} & \begin{minipage}[t]{0.12\columnwidth}\centering
-1.53\strut
\end{minipage} & \begin{minipage}[t]{0.14\columnwidth}\centering
0.1259\strut
\end{minipage}\tabularnewline
\begin{minipage}[t]{0.24\columnwidth}\centering
\textbf{factor(EDUC)3}\strut
\end{minipage} & \begin{minipage}[t]{0.14\columnwidth}\centering
-0.1969\strut
\end{minipage} & \begin{minipage}[t]{0.16\columnwidth}\centering
0.1498\strut
\end{minipage} & \begin{minipage}[t]{0.12\columnwidth}\centering
-1.314\strut
\end{minipage} & \begin{minipage}[t]{0.14\columnwidth}\centering
0.1888\strut
\end{minipage}\tabularnewline
\begin{minipage}[t]{0.24\columnwidth}\centering
\textbf{factor(EDUC)4}\strut
\end{minipage} & \begin{minipage}[t]{0.14\columnwidth}\centering
-0.05213\strut
\end{minipage} & \begin{minipage}[t]{0.16\columnwidth}\centering
0.1641\strut
\end{minipage} & \begin{minipage}[t]{0.12\columnwidth}\centering
-0.3176\strut
\end{minipage} & \begin{minipage}[t]{0.14\columnwidth}\centering
0.7508\strut
\end{minipage}\tabularnewline
\begin{minipage}[t]{0.24\columnwidth}\centering
\textbf{CIG}\strut
\end{minipage} & \begin{minipage}[t]{0.14\columnwidth}\centering
0.01946\strut
\end{minipage} & \begin{minipage}[t]{0.16\columnwidth}\centering
0.004191\strut
\end{minipage} & \begin{minipage}[t]{0.12\columnwidth}\centering
4.644\strut
\end{minipage} & \begin{minipage}[t]{0.14\columnwidth}\centering
3.424e-06\strut
\end{minipage}\tabularnewline
\begin{minipage}[t]{0.24\columnwidth}\centering
\textbf{CHOL}\strut
\end{minipage} & \begin{minipage}[t]{0.14\columnwidth}\centering
0.002371\strut
\end{minipage} & \begin{minipage}[t]{0.16\columnwidth}\centering
0.001127\strut
\end{minipage} & \begin{minipage}[t]{0.12\columnwidth}\centering
2.105\strut
\end{minipage} & \begin{minipage}[t]{0.14\columnwidth}\centering
0.0353\strut
\end{minipage}\tabularnewline
\begin{minipage}[t]{0.24\columnwidth}\centering
\textbf{BMI}\strut
\end{minipage} & \begin{minipage}[t]{0.14\columnwidth}\centering
0.006714\strut
\end{minipage} & \begin{minipage}[t]{0.16\columnwidth}\centering
0.01264\strut
\end{minipage} & \begin{minipage}[t]{0.12\columnwidth}\centering
0.5313\strut
\end{minipage} & \begin{minipage}[t]{0.14\columnwidth}\centering
0.5952\strut
\end{minipage}\tabularnewline
\begin{minipage}[t]{0.24\columnwidth}\centering
\textbf{GLUC}\strut
\end{minipage} & \begin{minipage}[t]{0.14\columnwidth}\centering
0.007186\strut
\end{minipage} & \begin{minipage}[t]{0.16\columnwidth}\centering
0.001674\strut
\end{minipage} & \begin{minipage}[t]{0.12\columnwidth}\centering
4.293\strut
\end{minipage} & \begin{minipage}[t]{0.14\columnwidth}\centering
1.766e-05\strut
\end{minipage}\tabularnewline
\bottomrule
\end{longtable}

(Dispersion parameter for binomial family taken to be 1 )

\begin{longtable}[]{@{}cc@{}}
\toprule
\endhead
\begin{minipage}[t]{0.27\columnwidth}\centering
Null deviance:\strut
\end{minipage} & \begin{minipage}[t]{0.37\columnwidth}\centering
3121 on 3657 degrees of freedom\strut
\end{minipage}\tabularnewline
\begin{minipage}[t]{0.27\columnwidth}\centering
Residual deviance:\strut
\end{minipage} & \begin{minipage}[t]{0.37\columnwidth}\centering
2759 on 3646 degrees of freedom\strut
\end{minipage}\tabularnewline
\bottomrule
\end{longtable}

\begin{enumerate}
\def\labelenumi{\alph{enumi})}
\setcounter{enumi}{2}
\item
\end{enumerate}

Walds tests: SBP:

Hypothesis: H0:β1 = 0 H1:β1 ≠ 0

test statistic: 5.238

p-value: 1.63e-07

given the fact that the p-value for SBP is significantly smaller than
any significance level in consideration we have enough evidence to
suggest that B1 is significantly different from 0 and therefore SBP is
statistically significant as a predictor of a 10 year risk of Coronary
Heart disease.

DBP:

Hypothesis: H0:β2 = 0 H1:β2 ≠ 0

Test statistic: -0.438

p-value: 0.6612

Considering that the p-value for DBP is quite large and it exceeds any
significance level, a conclusion can be formed that B2 is not
significantly different from 0 which means that the effect of DBP does
not appear to be a statistically significant for 10 year risk of
coronary heart disease.

\begin{enumerate}
\def\labelenumi{\alph{enumi})}
\setcounter{enumi}{3}
\tightlist
\item
  an increase of 1mm Hg in SBP is associated with an estimated
  multiplicative change of exp(0.018249) as shown below. The resulting
  figure corresponds to a roughly 2\% increase in the odds of having a
  10 year risk of future coronary heart disease.
\end{enumerate}

\begin{Shaded}
\begin{Highlighting}[]
\KeywordTok{exp}\NormalTok{(}\FloatTok{0.018249}\NormalTok{)}
\end{Highlighting}
\end{Shaded}

\begin{verbatim}
## [1] 1.018417
\end{verbatim}

\begin{Shaded}
\begin{Highlighting}[]
\KeywordTok{kable}\NormalTok{(}\KeywordTok{exp}\NormalTok{(}\KeywordTok{confint.default}\NormalTok{(logistic, }\DataTypeTok{parm =} \StringTok{"SBP"}\NormalTok{)), }\DataTypeTok{digits =} \DecValTok{3}\NormalTok{)}\OperatorTok
\StringTok{  }\KeywordTok{kable_styling}\NormalTok{()}
\end{Highlighting}
\end{Shaded}

\begin{table}[H]
\centering
\begin{tabular}{l|r|r}
\hline
  & 2.5 \% & 97.5 \%\\
\hline
SBP & 1.011 & 1.025\\
\hline
\end{tabular}
\end{table}

\begin{enumerate}
\def\labelenumi{\alph{enumi})}
\setcounter{enumi}{4}
\item
\end{enumerate}

\begin{Shaded}
\begin{Highlighting}[]
\NormalTok{hd.complete}\OperatorTok{$}\NormalTok{SBP_CAT <-}\StringTok{ }\KeywordTok{factor}\NormalTok{(hd.complete}\OperatorTok{$}\NormalTok{SBP_CAT, }\DataTypeTok{levels =} \KeywordTok{c}\NormalTok{(}\StringTok{"normal"}\NormalTok{, }\StringTok{"elevated"}\NormalTok{, }\StringTok{"High Stage 1"}\NormalTok{, }\StringTok{"High Stage 2"}\NormalTok{, }\StringTok{"hypertensive crisis"}\NormalTok{))}
\NormalTok{logistic2 <-}\StringTok{ }\KeywordTok{glm}\NormalTok{(HD_RISK }\OperatorTok{~}\StringTok{ }\KeywordTok{factor}\NormalTok{(SBP_CAT) }\OperatorTok{+}\StringTok{ }\NormalTok{DBP }\OperatorTok{+}\StringTok{ }\KeywordTok{factor}\NormalTok{(SEX) }\OperatorTok{+}\StringTok{ }\NormalTok{AGE }\OperatorTok{+}\StringTok{ }\KeywordTok{factor}\NormalTok{(EDUC) }\OperatorTok{+}\StringTok{ }\NormalTok{CIG }\OperatorTok{+}\StringTok{ }\NormalTok{CHOL }\OperatorTok{+}\StringTok{ }\NormalTok{BMI }\OperatorTok{+}\StringTok{ }\NormalTok{GLUC, }\DataTypeTok{family =}\NormalTok{ binomial, }\DataTypeTok{data =}\NormalTok{ hd.complete)}
\KeywordTok{summary}\NormalTok{(logistic2)}
\end{Highlighting}
\end{Shaded}

\begin{verbatim}
## 
## Call:
## glm(formula = HD_RISK ~ factor(SBP_CAT) + DBP + factor(SEX) + 
##     AGE + factor(EDUC) + CIG + CHOL + BMI + GLUC, family = binomial, 
##     data = hd.complete)
## 
## Deviance Residuals: 
##     Min       1Q   Median       3Q      Max  
## -1.7466  -0.6025  -0.4283  -0.2843   2.8458  
## 
## Coefficients:
##                                     Estimate Std. Error z value Pr(>|z|)
## (Intercept)                        -7.790357   0.696157 -11.191  < 2e-16
## factor(SBP_CAT)elevated             0.234938   0.161846   1.452 0.146609
## factor(SBP_CAT)High Stage 1         0.189517   0.177903   1.065 0.286746
## factor(SBP_CAT)High Stage 2         0.592431   0.184339   3.214 0.001310
## factor(SBP_CAT)hypertensive crisis  1.155536   0.298192   3.875 0.000107
## DBP                                 0.006231   0.005864   1.063 0.287955
## factor(SEX)1                        0.517430   0.108387   4.774 1.81e-06
## AGE                                 0.067004   0.006634  10.100  < 2e-16
## factor(EDUC)2                      -0.190932   0.122913  -1.553 0.120329
## factor(EDUC)3                      -0.215698   0.149497  -1.443 0.149068
## factor(EDUC)4                      -0.061899   0.164188  -0.377 0.706173
## CIG                                 0.019659   0.004184   4.699 2.62e-06
## CHOL                                0.002539   0.001126   2.256 0.024097
## BMI                                 0.006294   0.012617   0.499 0.617921
## GLUC                                0.007432   0.001667   4.457 8.29e-06
##                                       
## (Intercept)                        ***
## factor(SBP_CAT)elevated               
## factor(SBP_CAT)High Stage 1           
## factor(SBP_CAT)High Stage 2        ** 
## factor(SBP_CAT)hypertensive crisis ***
## DBP                                   
## factor(SEX)1                       ***
## AGE                                ***
## factor(EDUC)2                         
## factor(EDUC)3                         
## factor(EDUC)4                         
## CIG                                ***
## CHOL                               *  
## BMI                                   
## GLUC                               ***
## ---
## Signif. codes:  0 '***' 0.001 '**' 0.01 '*' 0.05 '.' 0.1 ' ' 1
## 
## (Dispersion parameter for binomial family taken to be 1)
## 
##     Null deviance: 3121.2  on 3657  degrees of freedom
## Residual deviance: 2768.2  on 3643  degrees of freedom
## AIC: 2798.2
## 
## Number of Fisher Scoring iterations: 5
\end{verbatim}

\begin{enumerate}
\def\labelenumi{\roman{enumi})}
\setcounter{enumi}{1}
\item
\end{enumerate}

\begin{Shaded}
\begin{Highlighting}[]
\NormalTok{elevated <-}\StringTok{ }\KeywordTok{exp}\NormalTok{(}\FloatTok{0.234938}\NormalTok{)}
\NormalTok{High1 <-}\StringTok{ }\KeywordTok{exp}\NormalTok{(}\FloatTok{0.189517}\NormalTok{)}
\NormalTok{High2 <-}\StringTok{ }\KeywordTok{exp}\NormalTok{(}\FloatTok{0.592431}\NormalTok{)}
\NormalTok{hypertensive <-}\StringTok{ }\KeywordTok{exp}\NormalTok{(}\FloatTok{1.155536}\NormalTok{)}

\NormalTok{elevated }
\end{Highlighting}
\end{Shaded}

\begin{verbatim}
## [1] 1.26483
\end{verbatim}

\begin{Shaded}
\begin{Highlighting}[]
\NormalTok{High1 }
\end{Highlighting}
\end{Shaded}

\begin{verbatim}
## [1] 1.208666
\end{verbatim}

\begin{Shaded}
\begin{Highlighting}[]
\NormalTok{High2 }
\end{Highlighting}
\end{Shaded}

\begin{verbatim}
## [1] 1.808379
\end{verbatim}

\begin{Shaded}
\begin{Highlighting}[]
\NormalTok{hypertensive}
\end{Highlighting}
\end{Shaded}

\begin{verbatim}
## [1] 3.175725
\end{verbatim}

SBP\_CAT:

elevated:

the p-value for the elevated level is 0.146609, this is quite large and
it exceeds any significance level which means we have evidence to
suggest that B1 is not significantly different from 0 which indicates
that an elevated SBP level is not statistically significant in
predicting a 10 year risk of coronary heart disease.

effect:

for the elevated level of SBP\_CAT, having Systolic blood pressure in
this range is associated with an estimated multiplicative change of
exp(0.234938) which corresponds to roughly a 26\% increase from a normal
level in the odds of having future coronary heart disease.

High stage 1:

the p-value for high stage 1 is 0.286746, this is also a large p-value
and is even larger than that of the elevated category. similarly to
elevated it exceeds all significance levels and therefore B1 is not
significantly different from 0 and in turn is not statistically
significant in predicting a 10 year risk of coronary heart disease.

effect:

for the High stage 1 level, having having Systolic blood pressure in
this range is associated with an estimated multiplicative change of
exp(0.189517) which corresponds to roughly a 20\% increase from the
normal level in the odds of having future coronary heart disease.

High stage 2:

the p-value for high stage 2 is 0.001310, this p-value is very small and
is alot smaller than any significance level, from this we can confirm
that there is evidence to suggest that B1 is significantly different
from 0 meaning that high stage 2 of SBP\_CAT is statistically
significant in predicting a 10 year risk of coronary heart disease.

effect:

for the High stage 2 level, having having Systolic blood pressure in
this range is associated with an estimated multiplicative change of
exp(0.592431) which corresponds to roughly a 217\% increase from the
normal level in the odds of having future coronary heart disease.

hypertensive crisis:

the p-value for hypertensive crisis is 0.000107, again this p-value is
very small and is alot smaller than any significance level, from this we
can confirm that there is evidence to suggest that B1 is significantly
different from 0 meaning that hypertensive crisis is statistically
significant in predicting a 10 year risk of coronary heart disease.

effect: for the hypertensive crisis level, having having Systolic blood
pressure in this range is associated with an estimated multiplicative
change of exp(1.155536) which corresponds to roughly a 80\% increase
from the normal level in the odds of having future coronary heart
disease. 1.155536

my results do not agree with the findings of Wu et al.~(2015), this is
because I have found that the risk of coronary heart disease is higher
with SBP from 120-129mm Hg sitting at 26\% (in respect to the reference
level) than the risk of coronary heart disease with SBP from 130 - 139mm
Hg sitting at 20\% (in respect to the reference level). this is a
contradiction to the findings of Wu but I am not entirely sure why this
is.

\begin{Shaded}
\begin{Highlighting}[]
\KeywordTok{kable}\NormalTok{(}\KeywordTok{exp}\NormalTok{(}\KeywordTok{confint.default}\NormalTok{(logistic2, }\DataTypeTok{parm =} \StringTok{"SBP_CAT"}\NormalTok{)), }\DataTypeTok{digits =} \DecValTok{3}\NormalTok{)}\OperatorTok
\StringTok{  }\KeywordTok{kable_styling}\NormalTok{()}
\end{Highlighting}
\end{Shaded}

\begin{table}[H]
\centering
\begin{tabular}{l|r|r}
\hline
  & 2.5 \% & 97.5 \%\\
\hline
SBP\_CAT & NA & NA\\
\hline
\end{tabular}
\end{table}

\begin{enumerate}
\def\labelenumi{\alph{enumi})}
\setcounter{enumi}{5}
\item
\end{enumerate}

\begin{Shaded}
\begin{Highlighting}[]
\KeywordTok{library}\NormalTok{(lmtest)}
\end{Highlighting}
\end{Shaded}

\begin{verbatim}
## Warning: package 'lmtest' was built under R version 3.6.3
\end{verbatim}

\begin{verbatim}
## Loading required package: zoo
\end{verbatim}

\begin{verbatim}
## Warning: package 'zoo' was built under R version 3.6.3
\end{verbatim}

\begin{verbatim}
## 
## Attaching package: 'zoo'
\end{verbatim}

\begin{verbatim}
## The following objects are masked from 'package:base':
## 
##     as.Date, as.Date.numeric
\end{verbatim}

\begin{Shaded}
\begin{Highlighting}[]
\KeywordTok{pander}\NormalTok{(}\KeywordTok{lrtest}\NormalTok{(logistic, logistic2))}
\end{Highlighting}
\end{Shaded}

\begin{longtable}[]{@{}ccccc@{}}
\caption{Likelihood ratio test}\tabularnewline
\toprule
\begin{minipage}[b]{0.07\columnwidth}\centering
\#Df\strut
\end{minipage} & \begin{minipage}[b]{0.11\columnwidth}\centering
LogLik\strut
\end{minipage} & \begin{minipage}[b]{0.06\columnwidth}\centering
Df\strut
\end{minipage} & \begin{minipage}[b]{0.10\columnwidth}\centering
Chisq\strut
\end{minipage} & \begin{minipage}[b]{0.16\columnwidth}\centering
Pr(\textgreater{}Chisq)\strut
\end{minipage}\tabularnewline
\midrule
\endfirsthead
\toprule
\begin{minipage}[b]{0.07\columnwidth}\centering
\#Df\strut
\end{minipage} & \begin{minipage}[b]{0.11\columnwidth}\centering
LogLik\strut
\end{minipage} & \begin{minipage}[b]{0.06\columnwidth}\centering
Df\strut
\end{minipage} & \begin{minipage}[b]{0.10\columnwidth}\centering
Chisq\strut
\end{minipage} & \begin{minipage}[b]{0.16\columnwidth}\centering
Pr(\textgreater{}Chisq)\strut
\end{minipage}\tabularnewline
\midrule
\endhead
\begin{minipage}[t]{0.07\columnwidth}\centering
12\strut
\end{minipage} & \begin{minipage}[t]{0.11\columnwidth}\centering
-1379\strut
\end{minipage} & \begin{minipage}[t]{0.06\columnwidth}\centering
NA\strut
\end{minipage} & \begin{minipage}[t]{0.10\columnwidth}\centering
NA\strut
\end{minipage} & \begin{minipage}[t]{0.16\columnwidth}\centering
NA\strut
\end{minipage}\tabularnewline
\begin{minipage}[t]{0.07\columnwidth}\centering
15\strut
\end{minipage} & \begin{minipage}[t]{0.11\columnwidth}\centering
-1384\strut
\end{minipage} & \begin{minipage}[t]{0.06\columnwidth}\centering
3\strut
\end{minipage} & \begin{minipage}[t]{0.10\columnwidth}\centering
9.35\strut
\end{minipage} & \begin{minipage}[t]{0.16\columnwidth}\centering
0.02498\strut
\end{minipage}\tabularnewline
\bottomrule
\end{longtable}

given that the p-value for this likelihood ration test is smaller than
any reasonable significance level with a value of 0.02498, we do have
sufficient evidence to suggest that we would prefer the second model
that uses SBP\_CAT over the first model that does not.

\begin{enumerate}
\def\labelenumi{\alph{enumi})}
\setcounter{enumi}{6}
\item
\end{enumerate}

\begin{Shaded}
\begin{Highlighting}[]
\KeywordTok{library}\NormalTok{(ResourceSelection)}
\end{Highlighting}
\end{Shaded}

\begin{verbatim}
## Warning: package 'ResourceSelection' was built under R version 3.6.3
\end{verbatim}

\begin{verbatim}
## ResourceSelection 0.3-5   2019-07-22
\end{verbatim}

\begin{Shaded}
\begin{Highlighting}[]
\KeywordTok{pander}\NormalTok{(}\KeywordTok{hoslem.test}\NormalTok{(hd.complete}\OperatorTok{$}\NormalTok{HD_RISK, logistic2}\OperatorTok{$}\NormalTok{fitted.values, }\DataTypeTok{g =} \DecValTok{10}\NormalTok{))}
\end{Highlighting}
\end{Shaded}

\begin{longtable}[]{@{}ccc@{}}
\caption{Hosmer and Lemeshow goodness of fit (GOF) test:
\texttt{hd.complete\$HD\_RISK,\ logistic2\$fitted.values}}\tabularnewline
\toprule
\begin{minipage}[b]{0.22\columnwidth}\centering
Test statistic\strut
\end{minipage} & \begin{minipage}[b]{0.06\columnwidth}\centering
df\strut
\end{minipage} & \begin{minipage}[b]{0.13\columnwidth}\centering
P value\strut
\end{minipage}\tabularnewline
\midrule
\endfirsthead
\toprule
\begin{minipage}[b]{0.22\columnwidth}\centering
Test statistic\strut
\end{minipage} & \begin{minipage}[b]{0.06\columnwidth}\centering
df\strut
\end{minipage} & \begin{minipage}[b]{0.13\columnwidth}\centering
P value\strut
\end{minipage}\tabularnewline
\midrule
\endhead
\begin{minipage}[t]{0.22\columnwidth}\centering
8.273\strut
\end{minipage} & \begin{minipage}[t]{0.06\columnwidth}\centering
8\strut
\end{minipage} & \begin{minipage}[t]{0.13\columnwidth}\centering
0.4073\strut
\end{minipage}\tabularnewline
\bottomrule
\end{longtable}

\begin{Shaded}
\begin{Highlighting}[]
\KeywordTok{pander}\NormalTok{(}\KeywordTok{hoslem.test}\NormalTok{(hd.complete}\OperatorTok{$}\NormalTok{HD_RISK, logistic2}\OperatorTok{$}\NormalTok{fitted.values, }\DataTypeTok{g =} \DecValTok{20}\NormalTok{))}
\end{Highlighting}
\end{Shaded}

\begin{longtable}[]{@{}ccc@{}}
\caption{Hosmer and Lemeshow goodness of fit (GOF) test:
\texttt{hd.complete\$HD\_RISK,\ logistic2\$fitted.values}}\tabularnewline
\toprule
\begin{minipage}[b]{0.22\columnwidth}\centering
Test statistic\strut
\end{minipage} & \begin{minipage}[b]{0.06\columnwidth}\centering
df\strut
\end{minipage} & \begin{minipage}[b]{0.13\columnwidth}\centering
P value\strut
\end{minipage}\tabularnewline
\midrule
\endfirsthead
\toprule
\begin{minipage}[b]{0.22\columnwidth}\centering
Test statistic\strut
\end{minipage} & \begin{minipage}[b]{0.06\columnwidth}\centering
df\strut
\end{minipage} & \begin{minipage}[b]{0.13\columnwidth}\centering
P value\strut
\end{minipage}\tabularnewline
\midrule
\endhead
\begin{minipage}[t]{0.22\columnwidth}\centering
17.36\strut
\end{minipage} & \begin{minipage}[t]{0.06\columnwidth}\centering
18\strut
\end{minipage} & \begin{minipage}[t]{0.13\columnwidth}\centering
0.4988\strut
\end{minipage}\tabularnewline
\bottomrule
\end{longtable}

\begin{Shaded}
\begin{Highlighting}[]
\KeywordTok{pander}\NormalTok{(}\KeywordTok{hoslem.test}\NormalTok{(hd.complete}\OperatorTok{$}\NormalTok{HD_RISK, logistic2}\OperatorTok{$}\NormalTok{fitted.values, }\DataTypeTok{g =} \DecValTok{30}\NormalTok{))}
\end{Highlighting}
\end{Shaded}

\begin{longtable}[]{@{}ccc@{}}
\caption{Hosmer and Lemeshow goodness of fit (GOF) test:
\texttt{hd.complete\$HD\_RISK,\ logistic2\$fitted.values}}\tabularnewline
\toprule
\begin{minipage}[b]{0.22\columnwidth}\centering
Test statistic\strut
\end{minipage} & \begin{minipage}[b]{0.06\columnwidth}\centering
df\strut
\end{minipage} & \begin{minipage}[b]{0.13\columnwidth}\centering
P value\strut
\end{minipage}\tabularnewline
\midrule
\endfirsthead
\toprule
\begin{minipage}[b]{0.22\columnwidth}\centering
Test statistic\strut
\end{minipage} & \begin{minipage}[b]{0.06\columnwidth}\centering
df\strut
\end{minipage} & \begin{minipage}[b]{0.13\columnwidth}\centering
P value\strut
\end{minipage}\tabularnewline
\midrule
\endhead
\begin{minipage}[t]{0.22\columnwidth}\centering
17.64\strut
\end{minipage} & \begin{minipage}[t]{0.06\columnwidth}\centering
28\strut
\end{minipage} & \begin{minipage}[t]{0.13\columnwidth}\centering
0.9349\strut
\end{minipage}\tabularnewline
\bottomrule
\end{longtable}

the Hosmer-Lemeshow tests for the second logistic regression model which
included SBP\_CAT show that as the amount of groups increases from 10 to
20 to 30 the p-value also increases which helps to show less and less
evidence against the current model being tested meaning that these
values appear to support the model being a reasonable fit for the
observed data.

this test does show that in reard to the number of groups it does apeear
to be better to have less groups as opposed to more.

Question 3:

\begin{enumerate}
\def\labelenumi{\alph{enumi})}
\item
\end{enumerate}

\begin{Shaded}
\begin{Highlighting}[]
\KeywordTok{library}\NormalTok{(MASS)}
\NormalTok{forward <-}\StringTok{ }\KeywordTok{stepAIC}\NormalTok{(}\KeywordTok{glm}\NormalTok{(HD_RISK }\OperatorTok{~}\StringTok{ }\DecValTok{1}\NormalTok{, }\DataTypeTok{data =}\NormalTok{ hd.complete, }\DataTypeTok{family =} \StringTok{"binomial"}\NormalTok{), }\DataTypeTok{scope =} \KeywordTok{list}\NormalTok{(}\DataTypeTok{upper =}  \OperatorTok{~}\StringTok{ }\NormalTok{SEX }\OperatorTok{+}\StringTok{ }\NormalTok{AGE }\OperatorTok{+}\StringTok{ }\KeywordTok{factor}\NormalTok{(EDUC) }\OperatorTok{+}\StringTok{ }\KeywordTok{factor}\NormalTok{(SMOKER) }\OperatorTok{+}\StringTok{ }\NormalTok{CIG }\OperatorTok{+}\StringTok{ }\KeywordTok{factor}\NormalTok{(BP_MED) }\OperatorTok{+}\StringTok{ }\KeywordTok{factor}\NormalTok{(STROKE) }\OperatorTok{+}\StringTok{ }\KeywordTok{factor}\NormalTok{(HYPER) }\OperatorTok{+}\StringTok{ }\KeywordTok{factor}\NormalTok{(DIAB) }\OperatorTok{+}\StringTok{ }\NormalTok{CHOL }\OperatorTok{+}\StringTok{ }\NormalTok{SBP }\OperatorTok{+}\StringTok{ }\NormalTok{DBP }\OperatorTok{+}\StringTok{ }\NormalTok{BMI }\OperatorTok{+}\StringTok{ }\NormalTok{HR }\OperatorTok{+}\StringTok{ }\NormalTok{GLUC, }\DataTypeTok{lower =} \OperatorTok{~}\DecValTok{1}\NormalTok{), }\DataTypeTok{direction =} \StringTok{"forward"}\NormalTok{, }\DataTypeTok{trace =} \OtherTok{FALSE}\NormalTok{)}
\NormalTok{forward}\OperatorTok{$}\NormalTok{anova}
\end{Highlighting}
\end{Shaded}

\begin{verbatim}
## Stepwise Model Path 
## Analysis of Deviance Table
## 
## Initial Model:
## HD_RISK ~ 1
## 
## Final Model:
## HD_RISK ~ AGE + SBP + SEX + CIG + GLUC + CHOL + factor(HYPER) + 
##     factor(STROKE)
## 
## 
##               Step Df   Deviance Resid. Df Resid. Dev      AIC
## 1                                     3657   3121.187 3123.187
## 2            + AGE  1 200.621281      3656   2920.566 2924.566
## 3            + SBP  1  65.007004      3655   2855.559 2861.559
## 4            + SEX  1  49.602275      3654   2805.956 2813.956
## 5            + CIG  1  19.987492      3653   2785.969 2795.969
## 6           + GLUC  1  19.116019      3652   2766.853 2778.853
## 7           + CHOL  1   4.081172      3651   2762.772 2776.772
## 8  + factor(HYPER)  1   2.985811      3650   2759.786 2775.786
## 9 + factor(STROKE)  1   2.286714      3649   2757.499 2775.499
\end{verbatim}

\begin{Shaded}
\begin{Highlighting}[]
\NormalTok{backward <-}\StringTok{ }\KeywordTok{stepAIC}\NormalTok{(}\KeywordTok{glm}\NormalTok{(HD_RISK }\OperatorTok{~}\StringTok{ }\NormalTok{AGE }\OperatorTok{+}\StringTok{ }\NormalTok{SBP }\OperatorTok{+}\StringTok{ }\NormalTok{SEX }\OperatorTok{+}\StringTok{ }\NormalTok{CIG }\OperatorTok{+}\StringTok{ }\NormalTok{GLUC }\OperatorTok{+}\StringTok{ }\NormalTok{CHOL }\OperatorTok{+}\StringTok{ }\KeywordTok{factor}\NormalTok{(HYPER) }\OperatorTok{+}\StringTok{ }
\StringTok{    }\KeywordTok{factor}\NormalTok{(STROKE), }\DataTypeTok{data =}\NormalTok{ hd.complete, }\DataTypeTok{family =} \StringTok{"binomial"}\NormalTok{), }\DataTypeTok{scope =} \KeywordTok{list}\NormalTok{(}\DataTypeTok{upper =}  \OperatorTok{~}\StringTok{ }\NormalTok{SEX }\OperatorTok{+}\StringTok{ }\NormalTok{AGE }\OperatorTok{+}\StringTok{ }\KeywordTok{factor}\NormalTok{(EDUC) }\OperatorTok{+}\StringTok{ }\KeywordTok{factor}\NormalTok{(SMOKER) }\OperatorTok{+}\StringTok{ }\NormalTok{CIG }\OperatorTok{+}\StringTok{ }\KeywordTok{factor}\NormalTok{(BP_MED) }\OperatorTok{+}\StringTok{ }\KeywordTok{factor}\NormalTok{(STROKE) }\OperatorTok{+}\StringTok{ }\KeywordTok{factor}\NormalTok{(HYPER) }\OperatorTok{+}\StringTok{ }\KeywordTok{factor}\NormalTok{(DIAB) }\OperatorTok{+}\StringTok{ }\NormalTok{CHOL }\OperatorTok{+}\StringTok{ }\NormalTok{SBP }\OperatorTok{+}\StringTok{ }\NormalTok{DBP }\OperatorTok{+}\StringTok{ }\NormalTok{BMI }\OperatorTok{+}\StringTok{ }\NormalTok{HR }\OperatorTok{+}\StringTok{ }\NormalTok{GLUC, }\DataTypeTok{lower =} \OperatorTok{~}\DecValTok{1}\NormalTok{), }\DataTypeTok{direction =} \StringTok{"backward"}\NormalTok{, }\DataTypeTok{trace =} \OtherTok{FALSE}\NormalTok{)}
\NormalTok{backward}\OperatorTok{$}\NormalTok{anova}
\end{Highlighting}
\end{Shaded}

\begin{verbatim}
## Stepwise Model Path 
## Analysis of Deviance Table
## 
## Initial Model:
## HD_RISK ~ AGE + SBP + SEX + CIG + GLUC + CHOL + factor(HYPER) + 
##     factor(STROKE)
## 
## Final Model:
## HD_RISK ~ AGE + SBP + SEX + CIG + GLUC + CHOL + factor(HYPER) + 
##     factor(STROKE)
## 
## 
##   Step Df Deviance Resid. Df Resid. Dev      AIC
## 1                       3649   2757.499 2775.499
\end{verbatim}

\begin{enumerate}
\def\labelenumi{\alph{enumi})}
\setcounter{enumi}{1}
\item
\end{enumerate}

\begin{Shaded}
\begin{Highlighting}[]
\CommentTok{#library(bestglm)}

\CommentTok{#predictors <- data.frame(SEX = hd.complete$SEX, AGE = hd.complete$AGE, EDUC = hd.complete$EDUC, SMOKER = hd.complete$SMOKER, CIG = hd.complete$CIG, BP_MED = hd.complete$BP_MED, STROKE = hd.complete$STROKE, HYPER = hd.complete$HYPER, DIAB = hd.complete$DIAB, CHOL = hd.complete$CHOL, SBP = hd.complete$SBP, DBP = hd.complete$DBP, BMI = hd.complete$BMI, HR = hd.complete$HR, GLUC = hd.complete$GLUC, y = hd.complete$HD_RISK)}

\CommentTok{#best.AIC <- bestglm(Xy = predictors, family = binomial, IC = "AIC", method = "exhaustive")}

\CommentTok{#$BestModels}
\end{Highlighting}
\end{Shaded}

\begin{enumerate}
\def\labelenumi{\alph{enumi})}
\setcounter{enumi}{2}
\item
\end{enumerate}

my computer is unable to perform the required subset selection in part
(b).


\end{document}
